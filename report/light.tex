\subsection{States, hashes, messages, representatives}

The light-node protocol states are also sets of messages,
each message being a triple $(c, v, j)$, where:
\begin{itemize}
    \item $c$ is a (proposed) consensus value
    \item $v$ identifies the sender of the message
    \item $j$, the justification, is the {\em set of hashes} of all messages
        of the protocol state in which the sender was at the time the message
        was sent
\end{itemize}

The fact that justification is a set of hashes allows for a simpler,
non-recursive definition. To begin, the type of hashes can be any
totally ordered set, a justification can be a list of hashes, and
a message is now just a triple:

\begin{coq}
Variable hash : Type.
Variable (about_H : `{StrictlyComparable hash}). 

Definition justification_type : Type := list hash. 

Definition message : Type := C * V * justification_type.
\end{coq}

The total order on hashes induces a total lexicographic order on
justifications, which can be extended to messages.

We therefore can work with sorted lists of hashes as representatives
for sets of hashes, reducing equality between justifications to
syntactic equality.

For states we prefer to define equality as set-equality, that is
double inclusion between the sets of messages representing states.

Assuming an injective function from messages to hashes,
\begin{coq}
Parameters (hash_message : message -> hash)
           (hash_message_injective : Injective hash_message).
\end{coq}    
we can recursively define a function \verb"hash_state" taking 
states to sorted lists of hashes, i.e., justifications, with the
property that two justifications are equal iff they belong to 
states which are equal as sets:

\begin{coq}
Lemma hash_state_injective : forall sigma1 sigma2,
  hash_state sigma1 = hash_state sigma2
  <->
  set_eq sigma1 sigma2.
\end{coq}    

This allows for the following inductive definition of protocol states:

\begin{coq}
Inductive protocol_state : state -> Prop :=
| protocol_state_nil : protocol_state state0
| protocol_state_cons : forall (j : state),
    protocol_state j ->
    forall (c : C),
      valid_estimate c j ->
      forall (v : V) (s : state),
        In (c, v, hash_state j) s ->
        protocol_state (set_remove compare_eq_dec (c, v, hash_state j) s) ->
        NoDup s ->
        not_heavy s ->
        protocol_state s.
\end{coq}

The above definition reads as:
\begin{itemize}
    \item a protocol state is either empty; or
    \item it is a non-heavy, non-duplicate state $s$ for which
        there exist a consensus value $c$, a sender $V$, and a state $j$ such that:
        \begin{itemize}
            \item $j$ is a protocol state
            \item $c$ is a consensus values which can be estimated by $j$
            \item $(c,v, \texttt{hash\_state}\  j)$ belongs to $s$
            \item The state obtained from $s$ by removing
                $(c,v, \texttt{hash\_state}\  j)$ is a protocol state.
        \end{itemize}
\end{itemize}


\subsection{Strong-non-triviality proofs}
Emphasis on explicating: 
\begin{enumerate}
	\item to the extent that it differs from the full node version, the strong non-triviality proof: atomic equivocation construction, pivotal validator proof, recursive atomic equivocation construction, overall proof sketch
\end{enumerate}
